\chapter{Introduction}

As of 2017, dental caries is the most prevalent disease \cite{Kassebaum2015}\cite{James2018} with more than 3.5 billion people infected. Despite the advancement of technology in the medical field, the prevalence does not decrease. This imposes a burden on health care in every country. In the US, more than six percent of total health care expenditures were targeted toward dental care in 2016\cite{Hung2020}.
\newline
Machine learning and especially neural networks have improved significantly over the last decade, even surpassing human-level performance on the ImageNet classification task in 2015\cite{He2015ICCV}. Since then, deep learning models' error rates on the ImageNet dataset are four-times smaller\cite{paperwithcode}. This significant improvement in deep learning led to its wast adoption across many fields, including medical imaging. Deep-learning models were even able to surpass human-level performance on certain tasks, such as breast cancer detection\cite{RodriguezRuiz2019} or arrhythmia detection\cite{Hannun2019}.
\newline
This work aims to develop a deep-learning-based model that would be able to detect dental caries in bitewing X-ray images. The position of every carious lesion is denited by a minimal bounding box around the lesion. This model aims to give a dentist an option to cross-check his decision regarding the presence of carious lesions in the X-ray image. Furthermore, detecting the position of carries directly from the image gives a dentist an option to save information about dental caries in digital form without his intervention. Having this information in digital form could help dentists communicate the problem to a patient by overlaying the position of dental caries over an X-ray image or providing him an option to save that information for monitoring lesion progression over time. Last but not least, software like this would be helpful in education, where dentistry students would be able to train their ability to recognize dental caries without the need of a human lector.


The structure of this thesis is as follows: TODO

Notes: human lector or tutor or teacher