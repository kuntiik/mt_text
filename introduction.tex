\chapter{Introduction}

As of 2017, dental caries is the most prevalent disease \cite{Kassebaum2015}\cite{James2018} with more than 3.5 billion people infected. Despite the advancement of technology in the medical field, the prevalence does not decrease. This imposes a burden on health care in every country. In the US, more than six percent of total health care expenditures were targeted toward dental care in 2016\cite{Hung2020}.
\newline
Machine learning and especialy neural networks have improved significantly over the last decade, even surpassing human-level performance on ImageNet classification task in 2015\cite{He2015ICCV}. Since then deep learning models error rates on ImageNet dataset are four-times smaller\cite{paperwithcode}. This significant improvment in deep-learning led to its wast adoption across many fields including medical imaging. Deep-learning models were even able to surpass human-level performance on certain tasks, such as breast cancer detection\cite{RodriguezRuiz2019} or arrhytmia detection\cite{Hannun2019}.
% \newline
% We decided to use deep learning for dental caries detection. Even thought several other publications trying to achieve similar goal are available TODO REF CHAPTER, we tried to apprach this problem differently. The goal is to detect and mark every tooth decay by minimal bounding box. New dataset containing 4000 images with carries lessions was developed for this purpose. 
\newline
The goal of this work is to develop deep-learning based model, that would be able to detect dental caries in bitewing X-ray images. Position of every carious lession is dented by minimal bounding box arround the lession. The purpose of this model is to give a dentist an option to cross-check his decission regarding pressence of carrious lession in X-ray image. Furthermore detecting position of carries directly from image gives a dentist an option to save information about dental caries in digital form withou his intervention. Having this information in digital form could help dentists to communicate the problem towards a pacient by overlaying the possition of dental caries over X-ray image or provides him an option to save those information for monitoring of lession progression over-time. Last but not least, software like this would be usefull in education, where dentistry students would be able to train their ability to recognize dental caries without the need of a human lector.

The structure of this thesis is as follows: In
