\chapter{Introduction}

As of 2017, dental caries is the most prevalent disease globally,\cite{Kassebaum2015}\cite{James2018} with more than 3.5 billion affected people.  Despite the advancement of technology in the medical field, the prevalence does not decrease, hence imposing a burden on health care in every country. In the US, more than six percent of total health care expenditures were targeted toward dental care in 2016\cite{Hung2020}.
\newline
Machine learning and especially neural networks have improved significantly over the last decade, surpassing human-level performance on the ImageNet classification task in 2015\cite{He2015ICCV}. Since then, deep learning models' error rates on the ImageNet dataset have become four times smaller\cite{paperwithcode}. This significant improvement in deep learning led to its wast adoption across many fields, including medical imaging. Deep-learning models exceeded human-level performance on specific tasks, such as breast cancer detection\cite{RodriguezRuiz2019} or arrhythmia detection\cite{Hannun2019}.
\newline
This work aims to develop a deep-learning-based model that will be able to detect dental caries in bitewing X-ray images. The position of every carious lesion is denoted by a minimal bounding box around the lesion. This model aims to give a dentist an option to cross-check his decision regarding the presence of carious lesions in the X-ray image. Furthermore, detecting the position of carries directly from the image gives the dentist an option to save information about dental caries in digital form without his intervention. Having this data in digital form could help dentists communicate the problem to a patient by overlaying the position of dental caries over an X-ray image or providing him an option to save that information for monitoring lesion progression over time. Last but not least, software like this would be helpful in education, where dentistry students would be able to train their ability to recognize dental caries without the need for a tutor.
\medskip


The structure of this thesis is as follows: In chapter \ref{chapter:medical_background} the medical background is introduced, describing human dentition and dental caries. Chapter \ref{chapter:theoretical_bg} introduces fundamentals of techniques used by this work. In chapter \ref{chapter:related_work} related work in automatic caries detection is discussed. Chapter \ref{chapter:dataset} describes the dataset that was created for the purpose of this thesis. Chapter \ref{chapter:project_structure} describes the reader, and the structure of the object detection framework, that we created and used for the purpose of caries detection. In chapter \ref{chapter:methods} we propose a solution to detection of dental caries. In chapter \ref{chapter:results} the results achieved by the proposed methods are presented. In chapter \ref{chapter:discusion} we discuss the results and methods that we used to obtain them. In the chapter \ref{chapter:conclusion} we summarise the work and suggest future improvements.


