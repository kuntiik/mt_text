
\ctusetup{
    % preprint = \ctuverlog,
    %	mainlanguage = english,
    %	titlelanguage = czech,
    mainlanguage = english,
    otherlanguages = {czech},
    title-czech = {Detekce zubních kazů z rentgenových snímků},
    title-english = {Dental caries detection from bitewing X-ray images},
    doctype = M,
    faculty = F3,
    department-czech = {Katedra počítačů},
    department-english = {Department of Computer Science},
    author = {Lukáš Kunt},
    supervisor = {prof. Dr. Ing. Jan Kybic },
    supervisor-address = {Biomedical imaging algorithms, FEE},
    supervisor-specialist = {},
    fieldofstudy-english = {Open informatics},
    subfieldofstudy-english = {Artificial inteligence},
    fieldofstudy-czech = {Otevřená informatika},
    subfieldofstudy-czech = {Umělá inteligence},
    keywords-czech = {detekce zubních kazů, konvoluční neuronové sítě, segmentace zubních výplní, ensemble, bitewing, rentgenový snímek},
    keywords-english = {dental caries detection, convolutional neural networks, dental restorations segmentation, ensemble, bitewing, X-ray image},
    day = 20,
    month = 5,
    year = 2022,
    specification-file = {zadani_prace.pdf},
    % specification-file = {images/IOU.jpg},
    front-specification = true,
    %	front-list-of-figures = false,
    %	front-list-of-tables = false,
    %	monochrome = true,
    %	layout-short = true,
}
\ctuprocess

\begin{abstract-english}
Dental caries is the most prevalent disease globally, with more than 3.5 billion people affected. The treatment of dental caries imposes a burden on health care in every country financially and timewise. Detection of the disease in its early stages can mitigate the impact on the cost of treatment and improve the patient's prognosis.

Bitewing X-ray imaging is the second most used method for dental caries detection after the visual-tactile method. Aproximal and an-early stage carious lesion can be easily overlooked by the visual-tactile exam, making the bitewing X-ray imaging very beneficial for early detection and a chance for recovery without the need for further dental treatment.

This Master's thesis addresses the problem of dental caries detection from bitewing images using convolutional neural networks. First, a dataset of 3889 bitewing images with 7257 annotated dental caries was created for the purpose of this thesis. We trained multiple architectures for object detection and compared their performance using it. In the end, we used an ensemble of models to obtain the best-performing model.


We have created a solution that can detect dental caries with a precision of 0.751 and a recall of 0.7. Furthermore, a second model for segmentation of dental restoration was created, achieving an IOU score of 0.676.
\end{abstract-english}

\begin{abstract-czech}
Zubní kaz je jedním z nejrozšířenějších onemocnění na světě postihující více než 3.5 miliard lidí. Léčba je náročná jak finančně, tak časově a zatěžuje zdravotnický systém ve všech zemích světa. Včasná detekce zubního kazu umožňuje tuto zátěž snížit a zlepšit pacientovu prognózu. 


Po detekci pohledem spojené s použitím zubařské sondy jsou bitewingové rentgenové snímky druhou nejvíce využívanou metodou pro diagnostiku zubního kazu. Časné a aproximální kariézní léze nejsou prvně zmíněnou metodou vždy spolehlivě diagnostikovány, což dává bitewingovým RTG snímkům značnou výhodu a šanci pro dřívější diagnostiku spojenou s možností zhojení léze bez nutnosti dalšího lékařského zásahu.


Tato diplomová práce se zabývá problémem detekce zubního kazu z bitewingových RTG snímků za použití konvolučních neuronových sítí. Pro účely této práce byl nejprve vytvořen dataset skládající se z 3889 bitewingových RTG snímků se 7257 anotovanými zubními kazy. Za jeho použití jsme natrénovali několik architektur pro detekci objektů a porovnali jejich výsledky. Nakonec jsme využili spojení modelů pro získání modelu s nejlepšími výsledky.


Vytvořili jsme řešení, které umožňuje detekci kazů s přesností 0.751 a citlivostí 0.7. Navíc byl vytvořen i druhý model pro segmentaci zubních výplní, který dosáhl IOU 0.676.
\end{abstract-czech}

\begin{declaration}
    I hereby declare that the presented work
    was developed independently and that I
    have listed all sources of information used
    within it in accordance with the Methodi-
    cal instructions for observing the ethical
    principles in the preparation of university
    thesis.
    \medskip
    Prague, May 20, 2022

\end{declaration}

\begin{thanks}
    Firstly, I would like to express gratitude to prof. Jan Kybic for supervision of this thesis and his willingness to help at any time.
    \medskip

    Secondly, I would like to thank MDDr. Tichý was always eager to help, and without his dedication to creating the dataset, we would not have achieved the results we did.
    \medskip

    Last but not least, I would like to thank my family for their support during my studies.
    \medskip

    I would like to further emphasize my gratitude to my girlfriend Anna for her undying moral and encouragement throughout the creation of this thesis.
\end{thanks}