
\ctusetup{
    % preprint = \ctuverlog,
    %	mainlanguage = english,
    %	titlelanguage = czech,
    mainlanguage = english,
    otherlanguages = {czech},
    title-czech = {Detekce zubních kazů z rentgenových snímků},
    title-english = {Dental caries detection from bitewing X-ray images},
    doctype = M,
    faculty = F3,
    department-czech = {Katedra počítačů},
    department-english = {Department of Computer Science},
    author = {Lukáš Kunt},
    supervisor = {prof. Dr. Ing. Jan Kybic },
    supervisor-address = {Biomedical imaging algorithms, FEE},
    supervisor-specialist = {},
    fieldofstudy-english = {Open informatics},
    subfieldofstudy-english = {Artificial inteligence},
    fieldofstudy-czech = {Otevřená informatika},
    subfieldofstudy-czech = {Umělá inteligence},
    keywords-czech = {detekce zubních kazů, konvoluční neuronové sítě, segmentace zubních výplní, ensemble, bitewing, rentgenový snímek},
    keywords-english = {dental caries detection, convolutional neural networks, dental restorations segmentation, ensemble, bitewing, X-ray image},
    day = 20,
    month = 5,
    year = 2022,
    specification-file = {zadani_prace.pdf},
    % specification-file = {images/IOU.jpg},
    front-specification = true,
    %	front-list-of-figures = false,
    %	front-list-of-tables = false,
    %	monochrome = true,
    %	layout-short = true,
}
\ctuprocess

\begin{abstract-english}
This work deals with object detection from the depth part of RGB-D data, which are produced by stereocamera Intel\textregistered{} Realsense$^{TM}$. Firstly, research of available methods for plane normal estimation from point clouds, picture segmentation and object detection is conducted. Based on this knowledge a program for detection of the surface normal and then precise location of on surface laying objects is proposed. This program is mainly based on RANSAC algorithm and PCA, which are combined with our aprior knowledge of the shapes of objects. Proposed programs are then evaluated on manualy labeled dataset, thereby it is validated, that they are capable of high accuracy object detection in real time.
\end{abstract-english}

\begin{abstract-czech}
Tato práce se zabývá detekcí objektů z hloubkové složky RGB-D dat, jejichž zdrojem je stereokamera Intel\textregistered{} Realsense$^{TM}$. Nejprve je provedena rešerše dostupných metod pro hledání normálového vektoru plochy v mračnech bodů, segmentaci obrazu a detekci objektů. Posléze je těchto znalostí využito k navržení vlastního programu na detekci normálového vektoru plochy a následnému přesnému určení polohy na ploše ležících objektů. Tyto programy vycházejí zejména z RANSAC algoritmu a PCA, které jsou kombinovány s naši apriorní znalostí tvaru objektů. Navržené programy jsou následně vyhodnoceny na manuálně označených datech, čímž je ověřeno, že jsou schopny přesné detekce v reálném čase.
\end{abstract-czech}

\begin{declaration}
    I hereby declare that the presented work
    was developed independently and that I
    have listed all sources of information used
    within it in accordance with the Methodi-
    cal instructions for observing the ethical
    principles in the preparation of university
    thesis.
    \medskip
    Prague, May 20, 2022

\end{declaration}

\begin{thanks}
    Firstly, I would like to epress gratitude to the prof. Jan Kybic

    Secondly, I would like to

    Last but not least, I would like to thank to my family for the support during the sutdies and creation of this thesis. I would like
\end{thanks}