\chapter{Discusion and further suggestions}
\section{Discusion}
% From the results in chapter \ref{chapter:results} and figures in appendix \ref{appendix:model_predictions}, we can see that the model is able to localize large portions of dental caries in the image, yet there is still room for improvement. From all the experiments, we observed the following phenomena:
% \begin{itemize}
%     \item In figure \ref{fig:yolov5_map_iou_thresholds}, we can see a sharp drop in the MAP values when the IOU threshold greater than 0.4 is chosen. The performance is present in both training and validation MAP curves. This seems to be related to the data rather than to the model. Also, it matches the expectations of Dr.Tichy, who said that in many cases, it is not crystal clear where precisely the tooth decay is located, and there is thus slight ambiguity in the image labeling.
%     \item From predictions in figures \ref{fig:pred_img1}, \ref{fig:pred_img2}, \ref{fig:pred_img3}, \ref{fig:pred_img4}, we can see that the model is incapable of precise predictions in the vicinity of dental restorations. This is a common pattern across the whole dataset and is probably caused by a wide variety of restorations shapes as well as by the low amount of similar images in the dataset.
%     \item The best performing YOLOv5 backbone was not the biggest one, but the 5m6 backbone, which is considered to be a medium-sized option. Even though the difference was negligible, it is still an unexpected result that the 5m6 backbone was able to outperform 5x6 backbone, even though it achieved ten percent better results than 5m6 when benchmarked on the MS COCO dataset \cite{glennjocher2020}. The inability to utilize the larger backbone can be caused by a wrong setup of the training pipeline or the low amount of data.
%     \item There was a drop in the model performance when selecting the image size of 1024 over 896. Even though the difference is subtle, we would expect the opposite. I personally do not know how the bigger image size could harm the performance.
%     \item Even though in MS COCO benchmark EfficientDet-D4 is a better performing model than YOLOv5-5l6, it has shown to be worse performing on our dataset. This could be caused by extremely low batch size of 1. The model has shown to be memory-intensive, and 48GB GPUs would be required to test bigger backbones with a batch size of 4.
% \end{itemize}
\section{Suggestions for further work}