\chapter{Medical background}

\section{Human teeth}
Human dentition is composed of two sets of teeth - primary and permanent. The primary teeth begin to erupt at age of 6 months of age and are replaced by permanent teeth by the approximate age of 13 years. Permanent dentition is composed of 32 teeth, which can be divided into 4 classes on the basis of function and form. Namely those classes are:

\subsubsection*{Incisors}
Total of 8 incisors teeth are found in permanent dentition. They are located near the entrance of oral cavity and they main purpose is to cut and shear food, they have alse vital role in human phonetics.

\subsubsection*{Canines}
Canies

\subsubsection*{Premolars}
Premolars shere functional charecteristics of canines and molars. They both seize and grind food.

\subsubsection*{Molars}
They main role is  crushing and grinding of food to dimensions appropriate for swallowing. Broad occlusal surfaces make them capable of this task. They are prone to caries, due to the presence of deep grooves that run accros the occlusal surface of the teeth and wide paint of cantact between adjacent molars. Both of those places are difficult to clean, making it a place where bacterias tend to accumulate.


\subsection{Structure of teeth}
Teeth is coposed of three structures: Enamel, pulp-denting complex and cementun. Picture of teeth structure is depicted in the figure

\subsection*{Enamel}
Highly mineralized crystalline structure, where more than 90\% of volume consits of minerals, is making enamel the thoughest part of teeth. It covers the crown of the tooth. Its thicknes varies from class of tooth to another, but in average it ranges from 2 to 3mm. Enamel does not have the ability of regeneration. The biggest threat for enamel are acidic condictions, which can cause its dissolution. This causes enamel to demineralize and when the cause of acidicity is not removed, the enamel starts to irreverisibly cavitate.

\subsection*{Pulp-Dentin comple}
Dental pulp is located in the pulp cavity in the tooth and its serves four functions: fromative, nutritive, sensory and reparative.
The pulp is circumscribed by dentin, which is formed by cells called odoblast. Their are considered to be part either pulp and dentin, because theri bodies cell bodies are in the pulp cavity, but thir cytoplasmic cell processes extent into minealized dentin. Because of those processes dentin is considered a living tissue, providing it abilities to regenerate and react to pathology stimuli, such as blocking advacement of carious lession by precipation of mineals in the affected area.

\subsection*{Cementum}
Cementum is covering anatomic roots of teeth. Its structure consists of approximately 50 \% of anorganic material and 50 \% of organic matter and water, making it slightly softer than dentin and far softer than enamel. It has ability to repair it self to a limited degree.

\section{Dental caries}

\subsection{Cause}
Dental caries is preventable chronic and biofilm-mediated diasese. The main cause is dental plaque (sometimes caled biofilm). Plaque is composed of bacteria, their by-products and water. The plaque adheres to tooth structure. Some bacteria in the plaque matabolize refine dietary carbohydrates and preduce organic acid by-products. Those acids if, if present in the biofilm for extended period periods of time, can lower the PH in the biofilm to bolow a critical threshold (5.5 for enamel, 6.2 for dentin). The low pH drives phosphate and calcium from the tooth into the biofil in attempt to reach equilibrium. This loss of calcium and phosphate in tooth is called deminealization. This process can be stoped and eventually reverted if the pH return to neutral and the relative concentration of soluble calcium and phosphate in the biofilm is higher than in the tooth.


\subsection{Classification of dental caries}
Dental caries are classified on multiple basies. The common ones are: Depth of the lession or lession activity. \newline
Commonly used classification shceme was proposed by Pitts \& Fyffe in 1988. Total of 4 categories are proposed, three for cavitated lesion and one for non-cavitated lesion.
\begin{itemize}
    \item \textbf{D0} Surface sound. No evidence of either treated and untreated caries.
    \item \textbf{D1} Initial Caries. No detectable loss of tooth substance. There can be staining in fissures or rough spots in enamel. Those spots do not catch the explorer.
    \item \textbf{D2} Enamel caries. Demonstrable loss of tooth substance. Chalky or crumble texture of the material within the cavity. No evidence, that cavitation has penetrated thru enamel layer into dentin.
    \item \textbf{D3} Caries of dentin. The floor or wall of the cavity is softened. The tip of explorer must enter a lesion with certanity.
    \item \textbf{D4} Pulpal involment. Deep cavity with probable invoment of the pulp. Probe should not be used to probe the pulp.
\end{itemize}

\subsection{Diagnosis}
Visual-tactile diagnosis is primary way how to inspect teeh and detect dental caries. The dentist uses mouth mirror and sharp probe to perform the examination. It is vital to dry teeth, since difference in the refractive index between sound and carious enamel is higher, when water is removed from the tissue. This increases the chance of spotting carious lession before its progression and cavitation.
Second most used approach is dental X-ray. In dentistry two types of X-rays are commonly taken. Periapical, which is used to asses overall state of dentistry and chance to detect dental-carries from them is small. On the other hand bitewing X-ray images are after visual-tactile diagnosis the most prominent type of diagnosis of dental caries. It has especialy prominent role in diagnosis of proximal surfaces. \newline
Among other less common diagnostic measures are:
\begin{itemize}
    \item Laser light-induced flurescence
    \item Digital imaging fiber-optic transillumination
    \item Electrical conducatance and impedance measurement
\end{itemize}

\subsection{Threatment}
Based on the progression of the lession and risk-profile of the pacient, Threatment is suggested. In some cases only increased oral higiene together with fluoride tooth paste is enough to stop the progression and remineralize enamel. The dentist can suggest application of a sealant to stop further progression of the lession. If this threatment is percieved as insuficient given the state of the lesion and the risk-profile of the patient, restoration of the tooth is required. This consists of the removal of any dental decay and filling the cavity with restorative material such as dental composite or amalgam.

\subsection{Epidemiology}
Untreated dental caries in permanent teeth is the most prevalent condition \cite{Kassebaum2015}. In 2010 more around 35\% of the global population was affected. The biggest prevalence was observed around age of 25. Sex of a person was not a significant factor in the statistcs. There was no noticable change in prevalence between the years 1990 and 2010 \cite{Kassebaum2015} \cite{Frencken2017}, this means that the technological improvment in dentistry had no effect on the prevalence.
Kassenbaum et al. suggests, that 42 new cases of tooth decay in primary and permanent teeth, will develop annually from 100 people followed up. This imposes a burden on health care systems. According to Huang at al. \cite{Hung2020} in the United States alone, the cost of dental care in 2016 was 0.1 trillion \$ out of total health care expenditure of 1.62 trillion \$.